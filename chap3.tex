\chapter{THE PACKET PIPELINE PROCESSING MODEL} \label{pipeline_model}

The Steve programming language uses a pipeline model for processing packets. A pipeline is a composition of two types of processing stages: decoding stages and table matching stages. Each stage performs a set of operations on a packet, known as actions, and decides where to move the packet next based on certain conditions that the packet meets. A packet can be moved to another processing stage or it can be sent out of the pipeline.

The pipeline is thus a state machine. Each processing stage denotes a state in the machine. Each state has a set of conditions which, when met, causes the packet to transition states. This can be represented as a graph where each processing stage is a node on the graph and each state transition is an edge connecting stages. 

Kinds of processing stages can be interleaved together in any order within the pipeline. This means that Steve is capable of supporting different packet processing paradigms found in other research such as POF and P4. With the Steve pipeline specification language a user can specify that a pipeline does:

\begin{enumerate}
\item \textbf{A full decode of the packet followed by a sequence of tables.} Packets coming to the pipeline have all necessary headers and fields are decoded and saved in the runtime context first. The packet is then dispatched to the first table in the pipeline. From there, matched flows within the tables dictate which table the packet is sent to next or which port the packet is forwarded to.
\item \textbf{A chain of partial decodes and table lookups.} Packets coming to the pipeline get partially decoded and dispatched to a table. The flow within that table could carry the packet to another table, another decoder, or forward it out of the network. The pipeline in this case in a chain of alternating sequences of decoding stages and table matching stages.
\item \textbf{Only decodes.} In some special cases, it may not even be necessary to go to a table matching stage. It may be possible to make a decision about the packet’s ultimate destination immediately upon evaluating a certain field within the packet using a simple conditional statement (if-statement, if-else statement, etc). Therefore, decoding stages also support the range of actions supported by flow entries, which can include outputting packets to a port or dropping it.
\end{enumerate}

Upon entering the pipeline, a packet must first go through at least one decoding stage before moving to the next processing stage. From there, the packet flows from one stage of the pipeline to the next. With each stage, certain conditions are evaluated which will determine where the packet must flow next. Finally, the port will exist the pipeline, either through a port or by being dropped and discarded.

