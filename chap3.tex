\chapter{THE PACKET PIPELINE PROCESSING MODEL} \label{pipeline_model}

A packet comes into a data plane via an ingress port. This packet must go through a series of processing stages where it is analysed, modified, and ultimately forwarded out through an egress port. These processing steps are what is known as a \textbf{pipeline}. 

The Steve programming language allows programmers to define their own pipeline applications. In the Steve processing model a pipeline is a composition of two types of processing stages: decoding stages and table matching stages. Each stage performs a set of operations on a packet, known as actions, and decides where to move the packet next based on certain conditions that the packet meets. A packet can be moved to another processing stage or it can be sent out of the pipeline.

The pipeline is thus a state machine. Each processing stage denotes a state in the machine. Each state has a set of conditions which, when met, causes the packet to transition states. This can be represented as a graph where each processing stage is a node on the graph and each state transition is an edge connecting stages. This is a valuable property as it allows the Steve programming language to analyse these pipelines and enforce certain logical, safety, and correctness guarantees about a user-defined pipeline.

Steve pipelines follow a run-to-complete model of execution. Once a packet enters the pipeline, each processing stage is consecutively applied to the packet until the programmer decides to forward or drop the packet.

\section{Decoding Stage} \label{decoder_desc}

When a packet is received on its ingress port, it is a chunk of raw, uninterpreted data. Before a packet can be processed and routed, its headers and fields must be decoded and extracted so that meaningful decisions can be made about what to do with the packet. The decoding stage is responsible for ensuring this happens.

Steve allows for programmers to specify \textit{how} and \textit{which} fields are extracted from packet headers. In other paradigms, the \textit{entire} packet is decoded from start to finish; all headers and all fields are extracted, then all fields are saved. This is what is considered a \textbf{full decode}. After this full decode, the decision making process on the packet begins using those saved fields. However, this method is inherently inefficient. Only certain fields and headers within a packet are ever really needed during the forwarding process. To compound this, different devices may care about different subsets of fields within a given packet. Decoding all of these fields does not make sense when only a smaller subset is ever necessary. 

Full decodes waste valuable processing time. Efficiency is important when dealing with networking equipment which has to processes between 10Gbps to 40Gbps. Decoding fields which are not needed is equivalent to wasting clock cycles on the CPU which translates to slower performance.

This inherent inefficiency is why Steve proposes the idea of a \textit{partial decode}. Unlike similar SDN-focused programming languages, Steve is designed to allow programmers to specify the extraction of only specific fields rather than an entire header. Though the specification may be verbose in some cases, it makes programmers think very carefully about which fields they need and which fields they do not.

Additionally, Steve proposes that not all headers need to be decoded. For example, if a networking application only needs to forward using MAC addresses, there is no reason to waste time extracting fields from IPv4 or IPv6 headers, and so on.
 
\subsection{Packet Context}

A stage often needs to use data created by prior stages. As a packet moves between stages, information such as the position and length of extracted fields and headers must be saved for recovery later in the pipeline. To save this information, Steve applications use a data structure called the packet \textbf{context}. Specifically, a Steve context saves the following data:

\begin{itemize}
\item The logical and physical port the packet arrived on.
\item The length of the packet frame.
\item A tunnel identifier.
\item The offset and length of a field in the packet.
\item The offset of a header in the packet.
\item An action set that can be written to.
\end{itemize}

Extracted fields and headers get saved in \textbf{binding environments} contained within the context. The term \textbf{environment} refers to a function that maps names (in this case field and header names) to their storage location during runtime \cite{compilers1}. The mapping of those storage locations to the values held there is known as \textbf{state}. There is a binding environment for fields and headers respectively.

Figure \ref{fg:ContextEnv} depicts the binding environment. The field binding environment is used to map fields to offset-length pairs where that field can be found in a packet. The packet header environment similarly maps headers to offsets where that header can be found in a packet. These mappings are known as \textit{bindings}.

Since any given packet can contain one or more of any field or header with the same name, the environments maintain a stack for every field and header. These stacks are what are called \textit{binding stacks}. By extension, this means an environment is actually a mapping of fields to binding stacks. When the value of a field is needed, the topmost binding on the binding stack shall be used to recover the state of that field.

The implementation of environments in Steve is a fixed-sized array where each element in the array is a fixed-sized binding stack. Each index in the array represents a unique field or header name extracted by the Steve application. The compiler is responsible for associating all unique fields extracted during the decoding stage to unique integer indices into the array. The same is done for all unique headers. This provides constant time lookup of field bindings without the overhead of hashing found in environments which use complex name mappings.

\begin{figure}
\centering
		\includegraphics[scale=0.75,natwidth=203,natheight=298]{context.png}
\caption{The binding environment inside a context used to store the length and offset of fields, or the offset of headers. On the left, fields one through sixteen represent the fields that can be extracted. Each field maintains a binding list (stack). Each element in the binding list is a binding which stores the offset and length where each instance of that field can be found in the packet. }
\label{fg:ContextEnv}
\end{figure}

Figure \ref{fg:ContextEnvWorking} demonstrates how data is stored in the context as it is being decoded. The example is a packet which contains an encapsulated IPv4 header commonly used in IP tunneling. In the ethernet decoding stage which extract the \texttt{src}, \texttt{dst}, and \texttt{type} field are extracted and stored in the context. Next we determine that IPv4 follows based on the \texttt{type} field. We extract IPv4 \texttt{dst} and \texttt{protocol}. The \texttt{protocol} field tells us we have another IPv4 header after the current one. We move to decode that and once again we extract IPv4 \texttt{dst} and \texttt{protocol}. Note how the new values of IPv4 are pushed on top of the binding stack. Any further usage of those fields will use the latest values extracted for those fields. Keep in mind that this means any usage of the first set of IPv4 \texttt{dst} and \texttt{protocol} must occur before the decoding of the second IPv4 header.\

\begin{figure}
TODO: Make an image for this.
\caption{A context environment in action during runtime.}
\label{fg:ContextEnvWorking}
\end{figure}

The \textbf{action set} is the other major data structure contained within the context. The action set is a set of actions which have been written to the context for deferred execution using a \texttt{write action} (see Section \ref{action_tut}). Upon reaching a designated egress port, the action set is executed right before the packet is output.

\section{Table Stage} \label{table_desc}

Table stages handle matching against extracted fields, i.e. classification, and perform a sequence of desired actions on like-classified packets. This is done through a mechanism known as a flow table \cite{openflow_spec}. 

A \textbf{flow table} is composed of a set of \textbf{flow entries}. A flow entry is composed of \textbf{match fields}, a \textbf{priority}, a set of \textbf{actions}, and miscellaneous additional data. Each table specifies a set of fields from packet headers that together make up a \textbf{key} for that table. Each flow entry must declare corresponding values, known as \textbf{match fields}, such that every flow entry in the table is uniquely identified by its match fields and its \textbf{priority}.

When a packet reaches a table matching stage, the fields comprising the table's key are extracted from the context. Lookup into the table retrieves all flow entries whose match fields can correctly match the field values from the packet. The flow entry whose priority is highest is selected. The \textbf{actions} of the flow entry are executed on the packet.

If no such flow entry matches against the packet's field values, the \textbf{miss case} flow entry is applied to the packet instead. Flow entries can be user defined. By default, a miss in a table results in the packet being dropped. Miss cases always have the lowest possible priority amongst flow entries and each match field can be considered a wildcard.

The mechanic of table matching is not distinctly different from those of relational or SQL tables. In fact, Frenetic, another packet processing language, uses SQL-inspired syntax to classify flows \cite{frenetic_paper1}. If we make this comparison, a flow table is analogous to a SQL table, the concept of a key is the same for both, and a flow entry is analogous to a tuple in a SQL table. Each packet and its fields constitute the actual "query" into the table.

\section{Pipeline Composition} \label{pipeline_comp_desc}

Kinds of processing stages can be interleaved together in any order within the pipeline. This means that Steve is capable of supporting different packet processing paradigms found in other research such as POF and P4. With the Steve pipeline specification language a user can specify that a pipeline does:

\begin{enumerate}
\item \textbf{A full decode of the packet followed by a sequence of tables.} Packets coming to the pipeline have all necessary headers and fields are decoded and saved in the runtime context first. The packet is then dispatched to the first table in the pipeline. From there, matched flows within the tables dictate which table the packet is sent to next or which port the packet is forwarded to.
\item \textbf{A chain of partial decodes and table lookups.} Packets coming to the pipeline get partially decoded and dispatched to a table. The flow within that table could carry the packet to another table, another decoder, or forward it out of the network. The pipeline in this case in a chain of alternating sequences of decoding stages and table matching stages.
\item \textbf{Only decodes.} In some special cases, it may not even be necessary to go to a table matching stage. It may be possible to make a decision about the packet’s ultimate destination immediately upon evaluating a certain field within the packet using a simple conditional statement (if-statement, if-else statement, etc). Therefore, decoding stages also support the range of actions supported by flow entries, which can include outputting packets to a port or dropping it.
\end{enumerate}

Upon entering the pipeline, a packet must first go through at least one decoding stage before moving to the next processing stage. From there, the packet flows from one stage of the pipeline to the next. With each stage, certain conditions are evaluated which will determine where the packet must flow next. Finally, the packet will exit the pipeline either through a port(s) or by being dropped and discarded.

