\chapter{THE PACKET PIPELINE PROCESSING MODEL} \label{pipeline_model}

The Steve programming language uses a pipeline model for processing packets. A pipeline is a composition of two types of processing stages: decoding stages and table matching stages. Each stage performs a set of operations on a packet, known as actions, and decides where to move the packet next based on certain conditions that the packet meets. A packet can be moved to another processing stage or it can be sent out of the pipeline.

The pipeline is thus a state machine. Each processing stage denotes a state in the machine. Each state has a set of conditions which, when met, causes the packet to transition states. This can be represented as a graph where each processing stage is a node on the graph and each state transition is an edge connecting stages.

\section{Decoding Stage}

When a packet is received, it is simply a chunk of raw, uninterpreted data. Before a packet can be processed and routed, its headers and fields must be decoded and extracted so that meaningful decisions can be made about what to do with the packet. The decoding stage is responsible for ensuring this happens.

Steve allows for programmers to specify \textit{how} and \textit{which} fields are extracted from packet headers. In other paradigms, the \textit{entire} packet is decoded from start to finish; then all fields are saved. This is known as a \textbf{full decode}. After this full decode, the decision making process on the packet begins using those saved fields. However, this method is inherently inefficient. Only certain fields and headers within a packet are ever really needed during the routing process. To compound this, different switches may care about different subsets of fields within a given packet. Decoding all of these fields does not make sense when only a smaller subset is ever necessary. 

Full decodes waste valuable processing time. Efficiency is important when dealing with networking equipment which has to processes between 10Gbps to 40Gbps. Decoding fields which are not needed is equivalent to wasting clock cycles on the CPU. This could translate to drastically slower performance.

This inherent inefficiency is why Steve proposes the idea of a \textit{partial decode}. Unlike other similar languages, Steve is designed to allow programmers to specify the extraction of only specific fields rather than an entire header. Though the specification may be verbose in some cases, it makes programmers think very carefully about which fields they need and which fields they do not.

\subsection{Packet Context}

As fields get extracted in the decoding stage, the offsets and lengths of those fields are saved into a data structure know as the packet \emph{context}. Each packet making its way through a packet processing pipeline is assigned its own context.


\section{Table Stage}

Table stages handle matching the extracted fields, i.e. classification, and perform actions on like-classified packets.

\section{Pipeline Composition}

Kinds of processing stages can be interleaved together in any order within the pipeline. This means that Steve is capable of supporting different packet processing paradigms found in other research such as POF and P4. With the Steve pipeline specification language a user can specify that a pipeline does:

\begin{enumerate}
\item \textbf{A full decode of the packet followed by a sequence of tables.} Packets coming to the pipeline have all necessary headers and fields are decoded and saved in the runtime context first. The packet is then dispatched to the first table in the pipeline. From there, matched flows within the tables dictate which table the packet is sent to next or which port the packet is forwarded to.
\item \textbf{A chain of partial decodes and table lookups.} Packets coming to the pipeline get partially decoded and dispatched to a table. The flow within that table could carry the packet to another table, another decoder, or forward it out of the network. The pipeline in this case in a chain of alternating sequences of decoding stages and table matching stages.
\item \textbf{Only decodes.} In some special cases, it may not even be necessary to go to a table matching stage. It may be possible to make a decision about the packet’s ultimate destination immediately upon evaluating a certain field within the packet using a simple conditional statement (if-statement, if-else statement, etc). Therefore, decoding stages also support the range of actions supported by flow entries, which can include outputting packets to a port or dropping it.
\end{enumerate}

Upon entering the pipeline, a packet must first go through at least one decoding stage before moving to the next processing stage. From there, the packet flows from one stage of the pipeline to the next. With each stage, certain conditions are evaluated which will determine where the packet must flow next. Finally, the port will exist the pipeline, either through a port or by being dropped and discarded.

