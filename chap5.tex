\chapter{User's Guide} \label{users_guide}

This chapter will explore the "anatomy" of Steve. It will dissect the semantics and limitations for each language feature. This chapter expects that the user has read the Tutorial and has a basic understanding of Steve. Some semantics from the Tutorial chapter will be repeated and expanded. 

This section shall elaborate on the conventional and illegal uses of the language. In many cases, the syntax used by the programmer has been abstracted away from much grittier details via "compiler magic". This chapter shall also elaborate on these inner mechanics.

Steve makes guarantees about the logical correctness and safety of pipeline stage composition. This section will detail how the Steve language enforces those guarantees and prove that these guarantees can be correctly enforced.

\section{Identifiers} \label{identifiers_guide}

An \textit{identifier} is an arbitrarily long sequence of characters. Supported characters include uppercase Latin letters (\texttt{A - Z}), lowercase Latin letters (\texttt{a - z}), digits (\texttt{0 - 9}), and underscores (\_). A valid identifier must begin with a non-digit character. Identifiers are case sensitive. In the Steve grammar, identifiers end with the suffix \textit{-id}.

Identifiers are subject to the following limitations:

\begin{itemize}
\item Identifiers which are keywords cannot be used for other purposes (see Section \ref{keyword_guide}).

\item Identifiers beginning with double underscores (\_\_) or an underscore followed by a capital letter (ex. \_F) are reserved by the compiler for internal identifiers.
\end{itemize}

Identifiers can be used as \textit{names} for \textit{entities}. An entity is a value, object, reference, function, layout, layout field, decoder, table, flow entry, port, event, extraction. In the Steve grammar, identifiers being used as names end with the suffix \textit{-name}.

Identifiers that name a variable, function, or port can be used as an expression. In this case, the identifier becomes an \textit{identifier expression}. In cases where identifier expressions refer to object declarations (see \ref{object_guide}) a \textit{reference} to \textit{value} conversion is applied (see Section \ref{reftoval_conv}.

\section{Keywords} \label{keyword_guide}

A number of identifiers in Steve are reserved as \textit{keywords}. The meaning and semantics of these identifiers cannot be changed. A list of Steve keywords can be found in Figure \ref{keywords_table}. 

\begin{figure} [ht]
{\ttfamily
\begin{tabular*}{\textwidth\noindent}{@{\extracolsep{\fill}} l l l l l}
bool   & break   & char    & continue & def  \\
if     & else    & foreign & int      & uint \\
return & struct  & this    & var      & while \\
match  & case    & layout  & decoder  & decode \\
start  & extract & as      & exact\_table & requires \\
miss   & Port    & goto    & output   & write \\
drop   & flood   & clear   & set      & insert \\
remove & into    & from    & event    & raise \\
in\_port & in\_phys\_port & all & controller & reflow \\
advance & egress & struct & char
\end{tabular*}
}
\caption{Steve reserved keywords. Note that Steve reserves the right to make any identifiers keywords in future versions.}
\label{keywords_table}
\end{figure}

\section{Field Name} \label{field_name_guide}

\textit{Field names} are special, \textit{qualified} names which refer to field declarations made within a layout declaration. Field names have the form:

\begin{minip}
\begin{grammar}
<field-name> ::=
<layout-id> \textbf{.} <field-id>
\alt <field-name> \textbf{.} <field-id>
\end{grammar}
\end{minip}

In the first form \grd{layout}{id}.\grd{field}{id}, or more succinctly \texttt{E1.E2}, \texttt{E1} must be a valid identifier to a layout declaration. \texttt{E2} must be found in the layout scope of \texttt{E1} using a qualified name lookup (see \ref{qlfd_lookup}).

In the second form \grd{field}{name}.\grd{field}{id}, or more succinctly \texttt{E1.E2}, \texttt{E1} must be a valid field name which refers to a field declaration whose type, \texttt{T}, is layout type (see \ref{layout_type_guide}). \texttt{E2} must be found in the layout scope of the layout declaration declaring type \texttt{T} using qualified name lookup.

In both forms, \texttt{E1} is referred to as the \textit{container} and \texttt{E2} is referred to as the \textit{contained field}.

The \textit{containing layout identifier} is the phrase used to mean the leftmost identifier in a field name.

Field names used as expressions become \textit{field access expressions} (see \ref{field_access_expr_guide}). Field access expressions may be used to refer to the value of the last extraction of that field name.

\section{Scope} \label{scope_guide}

Steve scope semantics borrow heavily from C++ scope semantics \cite{cpp_std}. \textit{Declarations} are used throughout program text to introduce \textit{names}. Names are \textit{identifiers} used to identify \textit{entities} (see \ref{identifiers_guide}). A \textit{name} is only \textit{valid} within parts of program text called the \textit{scope} of that name. A particular name is only considered valid if an \textit{entity} with that name can be found using an \textit{unqualified name lookup} (see \ref{unqlfd_lookup}). 

\subsection{Global Scope} \label{global_scope}

The outermost part of program text where declarations can be made is known as \textit{global scope}. All declarations made at global scope are said to be \textit{global declarations} and their names are said to be \textit{global names}.

Global names are valid at any point in the program. Steve does not require forward declarations. 

Two different declarations of the same name shall not be made at global scope. Any attempts to do this shall produce a compiler error.

\subsection{Block Scope} \label{block_scope}

Blocks are portions of program text which can have their own local declarations. The beginning of a block is delimited by the left-brace (\{) and the end of a block is delimited by the right-brace (\}). A declaration made within a block is in \textit{block scope} and is \textit{local} to that block.

Two different declarations of the same name cannot be made inside the same block. Any attempts to do this shall produce a compiler. 

Scopes can be nested. In this case, the inner scope is said to be the \textit{enclosed} scope, and the outer scope is said to be the \textit{enclosing} scope. The same name can be declared in the enclosing scope, and again in one or more enclosed scopes. The same is true for further nested scopes within the enclosed scope. If this happens, the scope of the outer declaration is its typical scope excluding program text of the enclosed scope. Unqualified name lookup (see \ref{unqlfd_lookup}) shall be used to unambiguously determine which declaration the name refers to. 

The statements \texttt{if}, \texttt{while}, \texttt{match} all implicitly introduce a block.

%Here, it is useful to talk about a concept called \textit{potential scope}, similarly found in C++ \cite{cpp_std}. The scope of a declaration is the same as its potential scope. The only exception is if there are two or more declarations of the same name in the same potential scope. In this case, the actual scope of a declaration is its potential scope excluding the potential scope of the inner declarative region.

Blocks found in the program text at global scope (i.e. function bodies, layout bodies, decoder bodies, etc) introduce block scopes which are nested in global scope.

For example, in the following example, the name \texttt{i} is declared twice. The scope of the first \texttt{i} is global scope and includes the entire example excluding the block between the first left-brace (\texttt{\{}) and the closing right-brace (\texttt{\}}). The scope of the second \texttt{i} begins immediately after its declaration and ends at the the closing right-brace (\texttt{\}})

\noindent\begin{minipage}{\linewidth}
\begin{lstlisting}
var i : int = 0;
def f() -> int
{
	var i : int = 1;
	var j : int = 2 + i;
	return j; // Result here shall be 3.
}
\end{lstlisting}
\end{minipage}

\subsection{Function Scope} \label{function_scope}

The body of a function declaration (see \ref{function_guide}) is its \grd{block}{statement} which introduces a block. All names introduced by a \grd{parameter}{declaration} in a function declaration's \grd{parameter}{sequence} have an effective scope starting at the beginning of the block.

\subsection{Decoder Scope} \label{decoder_scope}

The body of a decoder declaration (see \ref{decoder_guide}) is its \grd{block}{statement} which introduces a block. Declarations made inside this block are said to have \textit{decoder scope}.

\subsection{Table Scope} \label{table_scope}

A table declaration (see \ref{table_guide}) has a body given by \grd{table}{initializer} which introduces a block. Declarations made inside this block are said to be in \textit{table scope}.

All field names given by a \grd{key}{declaration} within \grdd{key}{declaration}{sequence} extends the scope of those field names into the table scope. Those field names are names declared via extract declaration (see \ref{extract_guide}) in a prior stage declaration of the pipeline. Field names declared via \grdd{key}{declaration} are valid and can be extended into the table scope if and only if \textit{requirements satisfaction} (see \ref{requirements_guide}) determines that field is satisfied. 

All field names given by the \grd{requires}{clause} extends the scope of those field names into the table scope. \textit{Requirement satisfaction} is used to check whether or not those names are valid and can be extended.

In the following example, we declare two decoders; \texttt{ipv4\_d} has extract declarations for field names \texttt{ipv4.type} and \texttt{ipv4.ttl}; \texttt{udp\_d} has extract declarations for \texttt{udp.dst\_port}.

\noindent\begin{minipage}{\linewidth}
\begin{lstlisting}
decoder ipv4_d(ipv4)
{
	extract ipv4.type;
	extract ipv4.ttl;
	// ...
	decode udp;
}

decoder udp_d(udp)
{
	extract udp.dst_port;
	// ...
	goto t1;
}
\end{lstlisting}
\end{minipage}

We go to table \texttt{t1} from \texttt{udp\_d}. Table \texttt{t1} has the key declarations \texttt{ipv4.type} and \texttt{udp.dst\_port} introduce those field names into table scope. What they are actually doing is \textit{extending} the scope of those names from the respective extract declarations into the scope of \texttt{t1}. The same \textit{scope extension} is applied to \texttt{ipv4.ttl}.

\noindent\begin{minipage}{\linewidth}
\begin{lstlisting}
exact_table t1(ipv4.type, udp.dst_port)
	requires(ipv4.ttl)
{
	{ 0x01, 80 } ->
	{
		// The ipv4.type field name is valid here.
		set ipv4.type = 0x02;
		// The udp.dst_port field name is valid here.
		set udp.dst_port = 88;
		// The ipv4.ttl field name is valid here.
		set ipv4.ttl = ipv4.ttl - 1;
	}
}
\end{lstlisting}
\end{minipage}

\subsection{Flow Entry Scope} \label{flow_scope}

A flow declaration has a body given by \grd{flow}{body} which introduces a block. Declarations made in this block are said to have \textit{flow entry scope}. If a flow declaration is part of a table declaration's \grd{table}{initializer}, the flow declaration has table scope. The flow entry scope is nested inside table scope.

If a flow declaration is part of an \texttt{insert} action, the body is treated as if it were nested inside the table scope of the table it is being inserted into. 

Names found in the \grd{properties}{block} follow regular scope semantics. That is, all names which are normally valid at the point where the \grd{properties}{block} appears in the program text are valid inside the \grd{properties}{block}. However, these names may not be valid for the \grd{flow}{body} in the case of an \texttt{insert} action.

In the following example, the name \texttt{x} is valid in the properties block, but not in the body of the flow.

\noindent\begin{minipage}{\linewidth}
\begin{lstlisting}
event e1 
{
	var x : int = 100;
	insert
	[timeout = x] // 'x' is valid here.
	{ x } -> // 'x' is also valid here. 
	{
		// Error: The name 'x' is not valid here.
		set eth.type = x + 1; 
		// However, the field name 'eth.type' is valid
		// because the flow body is treated as if nested
		// inside the scope of table 't1', whose key declaration
		// 'eth.type' makes the field name valid.
		flood; 
	}
	into t1;
}

exact_table t1(eth.type) { }
\end{lstlisting}
\end{minipage}

\subsection{Event Scope} \label{event_scope}

The body of an event declaration (see \ref{event_guide}) is a \grd{block}{statement} which introduces a block. All field names given by the \grd{requires}{clause} \textit{extends} the scope of those field names into the block. \textit{Requirement satisfaction} is used to check whether or not those names are valid and can be extended.

\subsection{Layout Scope} \label{layout_scope}

The body of a layout declaration (see \ref{layout_guide}) introduces a special layout block. Field declarations introduce field names into this layout block and are said to have \textit{layout scope}. Names of a field declaration can only be used as follows:

\begin{itemize}
\item Inside the scope of its layout.
\item After the dot-operator (\texttt{.}) applied to the name of its field in as part of either a field name (see \ref{field_name}) or a field access expression (see \ref{field_access_expr_guide}).
\end{itemize}

\subsection{Unqualified Name Lookup} \label{unqlfd_lookup}

Unqualified name lookup attempts to find the corresponding declaration for a name being used. Unqualified name lookup begins at the innermost block, before the name is used and works outward toward enclosing blocks. If the declaration is not found in any enclosing blocks, global scope is searched. The innermost declaration (e.g. the first one found) with that given name found is considered the corresponding declaration. 

If the name refers to one or more function declarations, and is being used as a function call, an \textit{overload set} is associated with the name. If this is the case, \textit{argument dependent lookup} is applied. The function declaration chosen shall be the one whose parameter types match the argument types used in the function call.

A name must be declared before being used. Any attempts to use a name before its declaration shall result in a failed unqualified name lookup.

Global names used at any point are considered valid regardless of the order with which they are declared in the program text. For example, the following usage of the name \texttt{i} in function \texttt{foo} refers to a variable declaration made after the function declaration. This is considered valid.

\noindent\begin{minipage}{\linewidth}

\begin{lstlisting}
// The name 'i' is used even though it is declared later
// in global scope.
def foo() -> int { return 3 + i; }
// The name 'i' is declared here.
var i : int = 0;
\end{lstlisting}

\end{minipage}

If unqualified lookup fails to find a corresponding declaration, the result is a compiler error.

\subsection{Qualified Name Lookup} \label{qlfd_lookup}

Qualified name lookup attempts to find the corresponding declaration for a name in a given block. The search is done only on the given block before the usage of the name and does not expand outward to enclosing blocks.

Qualified name lookup is most often used for looking up of names following the dot-operator, where the name lookup is limited to the scope of the layout.

If qualified lookup fails to find a corresponding declaration, the result is a compiler error.

\section{Conversions} \label{conversions_guide}

There are a number of type conversions in Steve which are all implicitly applied. These implicit conversions are applied to expressions in the order with which they are enumerate in this section.

\subsection{Reference to Value Conversion} \label{reftoval_conv}

Expressions of reference type can be converted to expressions of value type. An object of reference type stores an address to its data. A \textit{reference to value conversion} causes the data to be loaded from the address stored by the reference into a temporary. The value contained in the temporary is used for the operation in place of the expression of reference type.

An identifier to an object declaration is an identifier expressions whose type is a reference to the type of the object. When used in a situation where the \textit{value} of the object is needed, a reference to value conversion is applied. 

In the following example, the identifier expression \texttt{x} has type reference to integer. When used as part of an additive expression, the reference to value conversion is implicitly applied so both operands have type integer.

\noindent\begin{minipage}{\linewidth}
\begin{lstlisting}
var x : uint = 10;

// The reference to value conversion is
// implicitly applied on the identifier expression
// 'x' here.
x + 5;
\end{lstlisting}
\end{minipage}

\subsection{Integer Conversions} \label{int_conv}

Steve will convert the precision and signed/unsigned-ness of integers where necessary.

If an integer \texttt{I1} of type \texttt{T1} must be converted to an integer of type \texttt{T2}, the following conversions get applied in order.

\begin{itemize}
\item If \texttt{T1} is an unsigned integer and \texttt{T2} is a signed integer, then the type of \texttt{I1} is converted to a signed integer type with the same precision as \texttt{T1}.

\item If \texttt{T1} is a signed integer and \texttt{T2} is an unsigned integer, then the type of \texttt{I1} is converted to an unsigned integer type with the same precision as \texttt{T1}.

\item If the precision of \texttt{T1} is less than the precision of \texttt{T2}, then the type of \texttt{I1} is converted (promoted) to an integer type with the same precision as \texttt{T2}.

\item If the precision of \texttt{T1} is greater than the precision of \texttt{T2}, then the type of \texttt{I1} is converted (demoted) to an integer type with the same precision as \texttt{T2}.
\end{itemize}

\subsection{Port to Integer Conversions} \label{port_conv}

In some cases, namely comparison and equality operators, an identifier expression of reference to port type may be converted to have integer type. A port to integer conversion takes a port object and converts its value to an integer value. The integer value is equal to the port object's system assigned ID number. Port objects which have not been initialized or are invalid have an integer value \texttt{0}.

If an expression \texttt{e1} of reference to port type must be converted to type \texttt{T} of integer type, first a reference to value conversion is applied, then a port to integer conversion is applied.

For example, in the following case, a port to integer conversion is implicitly applied on the identifier expression \texttt{p1} to allow it to be compared with a literal expression of integer type.

\noindent\begin{minipage}{\linewidth}
\begin{lstlisting}
Port p1; // Uninitialized port object.
def foo() -> bool {
	// Implicit port to integer conversion.
	if (p1 == 0)
		return true;
}
\end{lstlisting}
\end{minipage}

\section{Objects} \label{object_guide}

An \textit{object} in a Steve program is an area of memory that has size, lifetime, type, and value. An object may also optionally be given a name.

Variables, ports, tables, and temporaries are objects. Objects are created in Steve by declarations of variables, ports, and tables. Objects may be created where temporary values are required.

\subsection{Storage Duration} \label{storage_guide}

The \textit{storage duration} of an object describes the point where an object's memory is allocated and the point where its memory is deallocated.

\begin{itemize}
\item \textbf{Automatic}. Automatic storage duration implies that an object is allocated at the beginning of the code block where it is declared and deallocated at the end of a code block.
\item \textbf{Pipeline automatic}. Pipeline automatic storage duration is a property of the context object (see \ref{context_guide}). Upon packet ingress, the context is allocated, upon egress the context is deallocated.
\end{itemize}

\subsection{Lifetime} \label{lifetime_guide}

The \textit{lifetime} of an object is either equivalent or a subset of its storage duration.

\begin{itemize}
\item \textbf{Automatic}. Automatic lifetime begins at the point where an object is \textit{initialized} and ends at the point where an object is de-initialized.
\item \textbf{Pipeline automatic}. Pipeline automatic lifetime is equivalent to pipeline automatic storage duration.
\end{itemize}

\section{Context Data Structure} \label{context_guide}

The storage duration and lifetime of a context is said to be \textit{pipeline automatic} (see \ref{lifetime_guide}). Management of the context's memory is exclusively and automatically handled by internal mechanics of the runtime system. This object is used exclusively by the internals of the program to keep track of information about a packet. 

A user is never given direct access to a context. Context objects are implicit, yet invisible within the scope of all pipeline processing stages (decoders, tables, and events). Certain expressions, declarations, and actions might access, affect, or otherwise modify the data of the context in limited and well-defined ways. 
\begin{itemize}
\item Expressions which affect a context are: field access expressions, in port expression, and in physical port expression.

\item Declarations which affect a context are: extract declarations, rebind declarations, and table declarations (indirectly).

\item All actions affect a context in some way.
\end{itemize}

\section{Pipeline Guarantees} \label{pipeline_checking_guide}

Logical correctness and safety guarantees are enforced by ensuring each Steve pipeline has three important properties: \textbf{progress}, \textbf{termination}, and \textbf{requirement satisfaction}. Any pipeline which does not have all three properties shall produce a compiler error. Pipelines that cannot ensure these three properties risk catastrophic crashes or undefined behaviour during runtime. This cannot happen on important networking devices.

\subsection{Pipeline to Pipeline Graph Conversion} \label{pipeline_graph}

These properties are checked by first converting a Steve program pipeline into a graph. Each property thus becomes a graph evaluation algorithm on the pipeline graph. To derive a pipeline graph from a Steve program, first all stages (decoders, tables, and events) are pulled from the Steve program. Each stage becomes a \textbf{node} on the pipeline graph. Table stages, specifically, become a node with edges to all of its contained flow entries which become independent nodes themselves. All inserted flow entries are also added with edges from the table they would be inserted into. Added flows cannot violate pipeline properties either, regardless of when they get added.

Every \texttt{decode}, \texttt{goto}, and \texttt{raise} action found in a stage (or flow entry within a table) causes an \textbf{edge} to be added between the stage and the destination specified by the action. This edge is added even if the action is encapsulated by a conditional statement (such as if-else or match) because its impossible to determine during compile time whether or not that edge is reachable during runtime. 

Every pipeline graph must have exactly one source node (the starting decoder), but can have multiple sinks.

\subsection{Progress} \label{progress_guide}

The \textbf{progress} property says that a packet always moves to a later stage in a pipeline and can never move, or risk moving, to a previously visited stage. Progress is the fundamental nature of the pipeline abstraction. This prevents packets from infinitely looping between the same stages more than once. 

The pipeline graph shall be a directed acyclic graph (DAG). A DAG, by definition, has no cycles and cannot infinitely recur between stages.

%First of all, table nodes are assigned a hidden, unique, incremented integer identifier in the order with which they are declared. For example, the first table declared in a Steve program is given the integer identifier 0, the second table declare is given the identifier 1, and so on. Conventionally, tables are expected to be declared in the order with which they are expected to occur in a pipeline.
%
%The \textbf{table identifiers} rule says that at no point in the pipeline can a packet reach a target table if it has already visited a table stage whose integer identifier is higher than or equal to that of the target table. In other words, a packet can only go forward through tables, never backwards. This is compliant with the OpenFlow specification for tables \cite{openflow_spec}. Because of this table identifiers rule, it is impossible for any number of table stages to form a cycle with each other.
%
%\textit{Proof.} A cycle can only be formed if the next table has either been visited, or is contained within a path to a table that has been visited. This can only happen if the next table has an identifier less than or equal to that of the current largest identifier visited. This property is enforced by the way tables are numbered. Since the table identifiers rule prevents a packet from being sent to these tables, it is thus impossible to form a cycle.

\subsection{Termination} \label{termination_guide}

\textbf{Termination} ensures that a packet must eventually be forwarded out of a pipeline or be explicitly dropped. Packet's cannot simply be "forgotten" by the pipeline, i.e. a packet cannot have finished processing without having the \texttt{output}, \texttt{flood}, or \texttt{drop} action applied to it. Because a packet's memory on the system gets deleted by the runtime environment upon egress, it will be "forgotten" and leaked if this property is not a guarantee. Enough leaked memory would eventually cause a device crash.

This property is enforced by a mechanism known as \textbf{terminator actions}. Of the actions discussed in Section \ref{action_tut}, the following are considered terminator actions: \texttt{decode}, \texttt{goto}, \texttt{output}, \texttt{flood}, and \texttt{drop}. Terminator actions immediately move a packet from the current stage to the next stage (or to egress processing). 

Decoders and every flow entry (including the miss case) inside a table, must have one guaranteed terminator action. There can be multiple terminator actions in a stage (for example terminators found inside if-else or match blocks), but there must be one that is logically guaranteed to be reachable during compile time. In other words, the guaranteed terminator must occur outside the scope of a conditional statement's block.

Paired with the \textbf{progress} principle, the pipeline can logically guarantee that a packet is always forwarded or dropped using the appropriate action. The proof is as follows.

\textit{Proof.} By the progress property, a packet can only move to a stage it has never been to before. By the termination principle, each stage must end with a terminator action. It is evident that eventually, a valid Steve program can no longer allow \texttt{goto} or \texttt{decode} actions at a given stage on a given path because they would have all been visited. The only terminator action valid at this point would be \texttt{output}, \texttt{flood}, or \texttt{drop}. Since a terminator must still occur in that given stage, this guarantees that the packet is explicitly forwarded or dropped.

Note that this property is not required of event stages. Events operate on copies of packets which are managed by the runtime environment and are thus not subject to the same risks of memory leakage.

\subsection{Requirement Satisfaction} \label{requirements_guide}

\textbf{Requirement Satisfaction} ensures that field access can only be done on extracted fields. 

Every node in a pipeline graph has a set of \textbf{productions} and \textbf{requirements}.

A \textbf{production} is a field that has been extracted or created by the node's respective stage. Currently only decoding stages are capable of have productions because they are the only ones which can extract fields. If a field were pushed (i.e. created) onto the packet, that would also constitute a production (Steve does not currently support this action).

Given any path \textit{P}, comprised of a set of nodes and edges needed to reach a node representing a stage \textit{V}, a \textbf{requirement} of \textit{V} is a field that must be a production of at least one node in the path \textit{P}. A path \textbf{satisfies} a requirement of \textit{V} if and only if this definition holds true.

In Steve, only tables, flow entries, and events can have requirements. Tables implicitly require all of their key fields. Both stages can explicitly state their requirements using the \texttt{\color{blue}requires} clause. A flow entry's requirements are implicitly that of its possessing table.

The requirements satisfaction property says that given any node \textit{V}, all requirements of \textit{V} must be satisfied by all paths leading to \textit{V}. If this property does not hold true, the result is a compiler error.

Remember that the objective is to prevent field access on fields that have not been extracted. Also remember that field access can only be done in certain cases. Firstly by a decoder if that field has been extracted by that decoder. Secondly by a table only if that field is part of the table's key or given by the table's \texttt{\color{blue}requires} clause. Thirdly by a table's flow entry whose requirements are implicitly the same. Fourthly by an event if that field is given by the event's \texttt{\color{blue}requires} clause.

\textit{Proof.} Field access can only be done in certain cases, all of which are cases where the requirements allow it. By the definition of the requirements satisfaction property all nodes in a pipeline graph must have their requirements satisfied, i.e. those fields must have been extracted along the path to that node, otherwise the result is a compiler error. Therefore, it is evident that it is impossible to compile a pipeline where field access is done on non-extracted fields.

\subsection{Depth First Search Graph Checking} \label{dfs_desc}

In order to produce the most complete error messages, the Steve compiler uses depth-first traversal with backtracking to evaluate all possible paths in a pipeline graph, see Algorithm \ref{alg:dfs}. The time complexity of finding all paths from source to a sink in a DAG is O($V^2$). Assuming we have \textit{S} number of sinks, and \textit{S} is at worst \textit{V}, the time complexity of finding all paths in a DAG is O($V^3$).

As the algorithm traverses a path in the graph, it accumulates a set of productions at each node. At each node, the node's requirements are checked against the accumulating set of productions to confirm the \textbf{requirement satisfaction} property holds true. Any node which fails immediately produces a compiler error and further traversal along that path stops.

At any point in a given path, if a node's edge is directed toward a previously visited node in the path, e.g. a cycle is found, traversal past that node immediately stops and a compiler error is generated warning about the error. This ensures that the \textbf{progress} property is met. The \textbf{termination} property is actually implicit as long as the progress property is met. 

\begin{algorithm}
 \caption{Depth-first traversal with backtracking used to check pipeline properties.}
 \label{alg:dfs}
 \begin{algorithmic}
 \State
 \State \textbf{Input}: Let \textit{G} be the pipeline graph. Let \textit{v} be a node in \textit{G}. Let \textit{p} be a set of productions.
 \State \textbf{Output}: Whether or not the current stage violates the progress, termination, or requirements satisfaction property. If any property fails, output a compiler error.
 \State 
 
 \Function{DFS}{$G, v, p$}
 	\State v.visited = true
 	\State p.push(v.productions)
 	\If{meetsRequirements(v, p)}
 		\For{\textbf{all} i \textbf{in} G.adjacentNodes(v)}
 			\If{(i.visited == false) $\land$ canProgress(v, i)}
 				\State \Call{DFS}{G, i, p}
 			\Else
 				\State \Return compiler error
 			\EndIf
 		\EndFor
	\Else 	
 		\State \Return compiler error
 	\EndIf
 	
 	\State v.visited = false \Comment{As we backtrack, we reset the visited property so we can come down this node again in different path.}
 	
 	\State p.pop(v.productions) \Comment{As we backtrack, we remove the productions of the node from the set of productions.}
 \EndFunction
 \end{algorithmic}
 
\end{algorithm}

\section{Declarations} \label{declaration_guide}

\textit{Declarations}, generally speaking, introduce an entity and a name for that entity. When using that name, the declaration is used to determine how to interpret that name.

For example, a variable declaration (see \ref{variable_guide}) introduces a variable and a name for that variable. When the name for that variable is used in program text, it is used to mean the variable itself.

A \textit{definition} in Steve is equivalent to a declaration in all but one case. The only case where a definition is distinct from a definition is when the \texttt{foreign} specifier (see \ref{foreign_spec}) is attached to an incomplete function declaration.

Declarations have the form:

\begin{minip}
\begin{grammar}
<declaration> ::=
[ <specifier-seq> ] <declaration>
\alt <global-declaration>
\alt <key-declaration>
\alt <extract-declaration>
\alt <rebind-declaration>
\alt <flow-declaration>
\alt <field-declaration>
\alt <parameter-declaration>
\alt <variable-declaration>

<global-declaration> :: =
<port-declaration>
\alt <layout-declaration>
\alt <decoder-declaration>
\alt <table-declaration>
\alt <event-declaration>
\alt <function-declaration>
\end{grammar}
\end{minip}

All declarations must occur within a scope. A declaration that declares a layout, decoder, table, event, or function causes a scope to be nested in global scope. That scope might also have scopes nested within it.

Certain declarations can only occur within certain scopes. Specifically, declarations which declare ports, layouts, decoders, tables, events, and functions must occur at global scope. These cannot occur within block scope. These are known as \textit{global} declarations.

Declarations may have a number of specifiers (see \ref{spec_guide}) attached to them, given by \grd{specifier}{seq}. Specifiers modify the semantics of declarations.

\section{Specifiers} \label{spec_guide}

Specifiers modify the semantics of a declaration. A sequence of specifiers may optionally appear at the beginning of a declaration. Not all specifiers are allowed on all declarations. Not all specifiers are allowed to appear together before the same declaration. Note that there is currently only one supported specifier. A number of other specifiers will eventually be introduced.

Specifiers have the following form:

\begin{grammar}
\singlespace
<specifier> ::= \textbf{foreign}
\end{grammar}

\subsection{Foreign Specifier} \label{foreign_spec}

The \texttt{foreign} specifier can only appear before \grd{function}{declaration}. This specifier says that a function has foreign linkage, that is, the function is not defined in this scope but rather in a different program. This function is said to be a \textit{foreign function}. The name of a foreign function is not mangled. Foreign functions cannot have a definition, that is, the \grd{function}{declaration} cannot have a \grd{block}{statement}. This is the only case where a declaration varies from a definition.

Foreign linkage is most often used to link against the C runtime library. For example, the following introduces the \texttt{puts} C-function into global scope. Function calls to \texttt{puts} will call the C-function with the same name.

\noindent\begin{minipage}{\linewidth}
\begin{lstlisting}
foreign def puts(char[]) -> int;

def main() -> int {
	puts("Hello world.");
}
\end{lstlisting}
\end{minipage}
 
\section{Layout Declaration} \label{layout_guide}

\textit{Layout declarations} are used to define the physical structure of packets in memory. A layout declaration declares a \textit{layout type}. Layout declarations have the following form:

\begin{minip}
\begin{grammar}
<layout-declaration> ::=
\textbf{layout} <layout-name> 
\textbf{\{}
	<field-declaration> +
\textbf{\}}

<field-declaration> ::=
<field-name> \textbf{:} <type> \textbf{;}
\end{grammar}
\end{minip}

Layout declarations shall contain one or more \textit{field declarations}, denoted in the grammar as \grd{field}{declaration}. Field declarations introduce names for fields into layout scope. The type of a field declaration, denoted by \gr{type} shall be scalar or layout type. \footnote{Though \texttt{bool} and \texttt{char} are valid scalar types and can appear as \gr{type}, they will result in a compiler error if the given field is needed as part of a table's key. This is a limitation that will be adjusted in later revisions.} Types of field declarations are used to specify the length of the given field. Because this is the extent of the usage, there is no reason to support more complex user-defined types here. 

Field declarations must occur in the order with which they appear in the an actual instance of a header which the layout represents. Incorrect ordering will result in incorrect extractions (see \ref{extract_guide}).

All field declarations declared within a layout declaration define a \textit{complete} layout. Field declarations may not be added to a layout elsewhere.

Though layout declarations declare layout types, objects of layout type can never be created and functions cannot have parameters or returns of layout type. The following is not legal in Steve. A layout type may only appear as the type of a field declaration in layout scope.

\noindent\begin{minipage}{\linewidth}
\begin{lstlisting}
layout l1 { ... }
var x : l1; // Illegal.
def foo(x : l1) -> l1 { ... } // Illegal.
\end{lstlisting}
\end{minipage}

\subsection{Why Objects of Layout Type can not Exist.}

Remember that layouts are not classes. The primary reason for this differentiation in Steve are dynamically-sized fields in packet headers. Headers potentially have fields whose length is predicated upon some value discovered during runtime. These fields are said to have \textit{dynamically-sized type}. Some examples of this are the \texttt{options} fields in IPv4, IPv6, and TCP headers. 

Consider that when objects of user-defined type (i.e. class type) are constructed, stack space must be allocated for them. The amount of space must be known during compilation. By including a member of DST in a class it effectively "taints" a class, making the class a user-defined DST. It is not possible to stack allocate an object which contains a member of DST because the amount of memory needed is unknown until runtime. It suddenly becomes impossible to create variables of that user-defined type, making the entire abstraction meaningless. Therefore, DSTs cannot be members of user-defined types. 

Such objects can only be heap allocated, which Steve does not currently support. The only exception is if memory is being allocated for an object of non-user-defined, dynamically-sized type. In this case, stack space can be allocated for this "scalar" DST, but access to it would happen through pointer addressing. The only language which supports user-defined DSTs, Rust, only allows it under very limited circumstance \cite{rust_dst_std}.

By extension, because layouts must contain fields of DST, Steve cannot allow layouts to behave like a user-defined type. Therefore, objects of layout type can never be created.
  
Layouts cannot contain member functions. It logically follows that since objects of layout type cannot exist, there is no justification for having member functions.

\section{Decoder Declarations} \label{decoder_guide}

Decoder declarations have the following form:

\begin{minip}
\begin{grammar}
\singlespace
<decoder-declaration> ::=
\textbf{decoder} <decoder-name> [\textbf{start}] 
\textbf{(} <layout-id> \textbf{)}
<block-statement>
\end{grammar}
\end{minip}

The identifier given by \grd{layout}{id} must name a valid layout declaration at global scope. This \grd{layout}{id} is known as the \textit{layout rule} of the decoder. Different decoder declarations may use the same layout rule. The layout rule is used to determine the \textit{current view} and \textit{implicit advances} generated by a decoder (see \ref{decoder_view}). Extract declarations (see \ref{extract_guide}) may only extract fields from this layout.

The optional \texttt{start} keyword shall occur on exactly one decoder declaration in a given program text. This decoder shall be considered the source (root) of the pipeline graph during pipeline checking. This decoder shall be the first decoder applied to a packet context after ingress.

The execution of a decoder is similar to that of a function. Each statement within \grd{block}{statement} is executed in turn. 

\subsection{Decoder View and Shifting} \label{decoder_view}

The \textit{current view} of a packet begins where the first byte of the current header being decoded lies in memory. The end of the current view is found at the position of the beginning plus the length of the current header. For headers whose lengths can be determined during compile time using the layout declaration refered to be the layout rule, the current view is said to be \textit{statically-sized}. For headers whose lengths must be determined by operations performed at runtime, the current view is said to be \textit{dynamically-sized}.

Steve uses this view mechanism to allow for partial decodes of headers. Without the ability to find fields relative to the beginning of views, decoding phases would be forced into full forward decoding, meaning for a field to be extracted, all fields prior to it would have to be extracted. This prevents us from receiving the gains of partial decode.

The context is used to maintain a positional index, known as the \textit{view index}, corresponding to the beginning of the current view. At first, this index begins at 0, corresponding to the first byte of the packet, and thus the first byte of the first header. Each decoder shifts the current view by the length of the current header before transitioning to the next stage, that is, the length of the current header is added to the view index upon the execution of a \texttt{decode} or a \texttt{goto} action (see \ref{decode_guide} \& \ref{goto_guide}).

If the view is statically-sized, the shift shall be implicitly emitted as code by the compiler by using the layout rule. This is known as an \textit{implicit advance}.

If the view is dynamically-sized, the length of the shift must be explicitly qualified using the \texttt{\color{blue}advance} clause. This is known as an \texttt{explicit advance}.

\section{Extract Declaration} \label{extract_guide}

Extract declarations have the following form:

\begin{grammar}
<extract-declaration> ::=
\textbf{extract} <field-name> \textbf{;}
\end{grammar}

An \textit{extract declaration} declares that a field with the specified \grd{field}{name} (see \ref{field_name}) is to be extracted by the decoder. The field name is in scope at the point of the extract declaration. The extracted field, or \textit{extraction}, is an object in memory whose name shall be \grd{field}{name}.

An extract declaration must be in decoder scope. Attempting to put an extract declaration in any other context shall result in a compiler error.

The containing layout identifier of \grd{field}{name} (see \ref{field_name_guide}) shall be the same layout identifier as the one given by the decoder's \textit{layout rule} (see \ref{decoder_guide}). A decoder shall not extract fields from layouts which are not it's layout rule. For example, the following is illegal:

\begin{code}
decoder eth_d(eth) {
	// Error: Illegal, this decoder does not extract ipv4.
	extract ipv4.dst; 
}
\end{code}

\subsection{Determining the Location and Length of Extracted Fields}

An extract declaration produces an extraction by executing a set of instructions on a packet which saves the extraction's \textit{location} and \textit{length} to the context's header environment (see \ref{context_desc}, \ref{context_guide}). Information gathered from \grd{field}{name} is used to calculate these two values. This process is completely opaque to the user, and is described below.

\begin{enumerate}
\item Discover the field declaration $f$ referred to by \grd{field}{name}. The length of the extracted field is calculated by a function $len(f)$. The result of $len(f)$ is the \textit{size} (see \ref{type_size}) of $f$'s type.

\item The \grd{field}{name} is has the form \texttt{E1.E2} where \texttt{E1} is the \textit{container} and \texttt{E2} is the \textit{contained field}. The set $L$ is the set of all field declarations preceding $f$ in its containing layout declaration. The function used to calculate the \textit{relative offset} of $f$ is given as $rel(E1, E2)$ and is defined as:

\begin{enumerate}

\item If \texttt{E1} is a \grd{layout}{id}, then the $rel(f) = \sum_{x \in L}{} len(x)$.

\item If \texttt{E1} is a \grd{field}{name}, \texttt{E1}'s container is denoted as \texttt{E1.c}, and \texttt{E1}'s contained field is denoted as \texttt{E1.d}, then $rel(f) = \sum_{x \in L}{} ( rel(E1.c, E1.d) + len(f) )$

\end{enumerate}

\item Given a field name of the form \texttt{E1.E2}, and the current view index (see \ref{decoder_view}), $i$, the \textit{absolute offset} of the field being extracted is this $i + rel(E1, E2)$.

\item The location of an extracted field is its absolute offset, that is, the number of bytes it is away from the beginning of the packet.
\end{enumerate}

\subsection{Extracting the Same Field More Than Once}

It is important to note that a decoder is only ever looking at one header of a packet at once. Using an extract declaration with the same field name more than once in the same decoder will result in the same instance of that field being extracted multiple times, which is completely redundant. Though legal, this should be avoided in most cases.

\subsection{Rebind Declaration} \label{rebind_guide}

At certain times, it may be convenient to extract a certain field, but \textit{alias} that field with a different name than the original. This can be done with the \textit{extract-as}, otherwise known as the \textit{rebind} declaration. A rebind declaration has the following form:

\begin{minip}
\begin{grammar}
<rebind-declaration> ::=
\textbf{extract} <field-name> \textbf{as} <field-name> \textbf{;}
\end{grammar}
\end{minip}

The rebind declaration requires two field names be given. The first \grd{field}{name}, or more succinctly \texttt{N1}, shall be known as the \textit{original field name}. The second \grd{field}{name}, or more succinctly \texttt{N2}, shall be known as the \textit{alias field name}. \texttt{N1} and \texttt{N2} must both be valid field names.

The field declaration named by \texttt{N1} shall have the same type as the field declaration named by \texttt{N2}. 

A rebind declaration causes \texttt{N1} and \texttt{N2} to be in scope. Both names are added to the decoder's productions (see \ref{requirements_guide}). The names \texttt{N1} and \texttt{N2} may be used as field names in either key declarations or requires clauses in tables and events later in the pipeline. 

\section{Table and Flow Declarations} \label{table_guide}

Table declarations have the following form:

\begin{minip}
\begin{grammar}
<table-declaration> ::=
\textbf{table} <table-name> \textbf{(} <key-declaration-sequence> \textbf{)} 
[ <requires-clause> ] <table-initializer>

<key-declaration> ::=
<layout-id> \textbf{.} <field-id>
\alt <key-declaration> \textbf{.} <field-id>
\alt \textbf{in\_port}
\alt \textbf{in\_phys\_port}

<requires-clause> ::=
\textbf{requires} \textbf{(} <field-name-sequence> \textbf{)}

<table-initializer> ::= \textbf{\{} [ <flow-declaration-list> ] \textbf{\}} 

<flow-declaration-list> ::= <flow-declaration> 
\alt <flow-declaration-list> <flow-declaration>
\end{grammar}
\end{minip}

Table declarations may cause table objects to be created. After loading a Steve generated application, the runtime system (see \ref{flowpath}) shall receive a number of table allocation requests from the Steve program corresponding to each table declaration made in the program text. These requests are made as part of load-time configuration (see \ref{config_guide}). Each request shall provide:

\begin{itemize}
\item A table name.
\item The set of key fields that comprise the table's key.
\item A maximum number of flow entries.
\end{itemize}

If tables with those properties already exist, the runtime system provides the Steve application with the already existing table. This can happen if multiple Steve applications are run on the same device.

The table's \textit{key fields} are a set of one or more fields which together compose a \textit{key}. A table's key is given by \grdd{key}{declaration}{sequence} in the grammar. Each \grd{key}{declaration} shall also be a valid field name, or be the keywords \texttt{in\_port} or \texttt{in\_phys\_port}. The type of \grd{key}{declaration} shall be the type of the field declaration named by the same field name, or shall be port type (see \ref{port_type_guide}) in the case of \texttt{in\_port} or \texttt{in\_phys\_port}.

The \grd{requires}{clause} is optional. Each field name within the \grd{requires}{clause} must be a valid field name. A field name here may have the same program text as a \grd{key}{declaration} in the same table declaration. There is no change in semantics when this happens. Field names given by \grdd{key}{declaration}{sequence} are implicitly required.

Each table has a set of \textit{flow entries} declared in Steve using a \grd{flow}{declaration} of the following form:

\begin{minip}
\begin{grammar}
<flow-decl> ::=
<properties-block>
\textbf{\{} <match-field-sequence> \textbf{\}} \textbf{-\textgreater} <flow-body>
\alt <miss-flow-declaration>

<match-field> ::= <expression>

<flow-body> :: \textbf{\{} <action> + \textbf{\}}

<properties block> ::=
\textbf{[} <property-sequence> \textbf{]}

<property> ::=
<property-kind> \textbf{=} <expr>

<property-kind> ::=
\textbf{timeout}
\alt \textbf{egress}

<miss-flow-decl> ::=
\textbf{miss} \textbf{-\textgreater} <flow-body>
\end{grammar}
\end{minip}

The \textit{match fields} of a flow entry, that is, values corresponding to a table's key fields, is given by \grdd{match}{field}{sequence}. The type of each \grd{match}{field} must be the same type as its corresponding \grd{key}{declaration}. The count of \grd{match}{field} in \grdd{match}{field}{sequence} shall be equal to the count of \grd{key}{declaration} in \grdd{key}{declaration}{sequence}.

All flow entries in the same table must be uniquely identifiable by their \textit{match fields} and their \textit{priority} (since we only support exact match tables, the priority is the same on all flow entries). If a flow entry being inserted into a table has the same value in each match field as an already existing flow entry in the table, the previous flow entry is removed and replaced by the new flow entry.

The set of \textit{initial flow entries}, that is, the flow entries which are inserted into the table during load-time configuration, are given as \grd{flow}{declarations} within \grd{table}{initializer}. Each initial flow entry is inserted with the order in which they are declared. There may only be one or less \grd{miss}{flow}{decl} shall appear here.

\section{Port Declaration} \label{port_guide}

A port declaration has the following form:

\begin{minip}
\begin{grammar}
<port-decl> ::=
\textbf{Port} <port-name> \textbf{;}

\end{grammar}
\end{minip}

\section{Variable Declarations} \label{variable_guide}

Variable declarations have the form:

\begin{minip}
\begin{grammar}
<variable-decl> ::=
\textbf{var} <variable-name> \textbf{:} <type> [ <variable-initializer> ] \textbf{;}

<variable-initializer> ::= \textbf{=} <expression>
\end{grammar}
\end{minip}

A variable declaration may have an optional \grd{variable}{initializer} following \gr{type}.  If this initializer is not given, the result is default initialization. Default initialization of scalar types shall result in their objects having a binary value of zero.

If \grd{variable}{initializer} is given, the type of \gr{expression} shall match the type given by \gr{type}, or there shall exist a conversion sequence (see \ref{conversions_guide}) from the type of \gr{expression} to the type given by \gr{type}. The value given by \gr{expression} shall be copied into the allocated object memory, that is, \textit{copy initialized}.

\section{Function Declarations} \label{function_guide}

\section{Identifier Expressions} \label{id_expr_guide}

\section{Field Access Expression} \label{field_access_expr_guide}

\section{Unary Expressions} \label{unary_expr_guide}

\section{Binary Expressions} \label{binary_expr_guide}

\section{Statements} \label{statements_guide}

\section{Actions} \label{action_guide}

\section{Decode Action} \label{decode_guide}

\section{Goto Action} \label{goto_guide}

\section{Insert Action} \label{insert_guide}

\section{Block Statements} \label{block_stmt_guide}

\section{Assignment Statements} \label{assign_stmt_guide}

\section{Type} \label{type_guide}

\subsection{Type Size} \label{type_size}

\section{Layout Type} \label{layout_type_guide}

\section{Scalar Type} \label{scalar_type_guide}

\section{Port Type} \label{port_type}

\section{Runtime Configuration} \label{config_guide}
