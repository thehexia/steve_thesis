\chapter{ Conclusion and Future Work} \label{ch:conclusion}

The goal of this research was to present a safe and efficient language for writing network applications for SDN switches. Though other languages already exist, Steve sought to improve upon them. First, Steve provides strict compile-time safety guarantees of user-defined packet processing pipelines.
Steve's type and constraints system make it almost impossible to write a network application which invokes undefined behavior.
Secondly, the Steve language expands past the basic pipeline specification of most languages and allows the programmer to define protocol independent control plane event handling as well.
With that being said, there are still issues we have yet to explore.

\subsection{Language Use Cases}

As we are not network programming specialists, we were never completely sure of which language features were needed to support rich network applications. We chose to use OpenFlow \cite{openflow_spec} as our guide post of what was, at minimum, required. From there, we expanded on it from other languages such as POF \cite{pof, pof_fis, pof_impl} and P4 \cite{pof, p42014, p4_spec}. Steve may expand as the need for new use cases grow.

\subsection{Portability}

Portability is another major issue closely related to use cases. We do not own the SDN hardware capable of running Steve due to cost. We are also unaware of how many SDN switches are actually fully capable of all the Steve features. At the time of writing, these devices are rather exotic and lack standardization. This is why we choose to target the interoperable ABI from the Freeflow rather than compiling for specific switch instructions.

First the virtual machine, Freeflow \cite{freeflow_software} or some alternative software, would have to be ported to the target device to expose the ABI which Steve depends on. Then the Steve compiler would have to be rewritten to target the appropriate processor architecture. We suspect this is rare because LLVM already supports the most common ones.

\subsection{Runtime ABI}

The runtime ABI is designed to alleviate issues with compiling to other platforms.
As long as there is the ABI exposes functions, Steve can be compiled to target them. The advantage is that the runtime environment worries about managing hardware acceleration, optimization, etc., rather than the compiled Steve application. In future work, we believe it is possible to provide optimized resources (such as TCAM memory) to the application opaquely through the runtime ABI.

\subsection{Language Safety}

We have proven how Steve enforces pipeline safety, but
because Steve focuses on being safe over everything else, the language enforces very strict rules about pipeline composition. However, some of these rules can be weakened depending on use cases.
Specifically, the constraint that pipelines cannot have cycles can. It is possible to loosen this restriction so that pipelines may have cycles, but only between decoders. This is useful for recursively decoding an indeterminate number of headers of the same kind (like VLAN). This would not cause an infinite cycle because the view would eventually shift off the end of the packet causing it to be dropped.

\subsection{Distributed Control}

It is possible that, rather than compiling one binary application, it is possible for the Steve compiler to produce two: one containing data plane entities, and another containing control plane event handlers. This would be similar to how ASP.NET web forms divide embedded C\# from web page content. The binary containing event handlers could be loaded by another machine and act as distributed controller. The message format could then be the context data structure wrapped by network protocols. Alternatively, since we can process any protocol, including OpenFlow, it is possible to process OpenFlow messages and locally and respond accordingly.

\section{Future Work}

Steve is far from a complete and mature language. It is still missing key components that other languages of its kind already support. Dynamically-sized fields, group tables, alternative table matching patterns (prefix, wildcard, etc), pushing and popping headers, packet construction, etc. On top of that, we have yet to determine a programming model for distributed control.

Pushing and popping headers as well as packet construction are the most interesting. This would allow Steve to become more than purely a packet processing language. The Steve programmer could hypothetically create new packets as a reactive response to certain network traffic.

Device introspection, that is, the ability of the application to learn things about the switch is missing.
It may be useful for an application to search for a specific port, or to learn about the device's MAC and IP addresses.
Deep packet introspection is similarly useful. Most SDN languages only focus on headers, but being able to look into a payload allows certain security applications to be written on the switch.


\section{Conclusion}

Being able to program every component of a network switch is a powerful thing. Just by changing out a program, the entire behavior of the device can change. This is an incredibly powerful thing. It allows enterprises to quickly reconfigure their entire network setup at little to no cost. It also expands the possibilities of how the Internet may one day evolve. As SDN aims to make switches more like general purpose computers, the Internet may too evolve into something better.
With that being said, SDN devices are far from being commonplace. As it stands, SDN is still trades speed for flexibility, making enterprises hesitant to adopt it. 

Steve is just an infant step towards this potential future. This system addresses what a protocol independent, safe, and efficient SDN programming environment looks like. This research provides a solid foundation of language features that must exist to implement network applications that can hopefully expand in the future.