\chapter{BACKGROUND RELATED WORK}
\label{ch:related}

\section{Background}
\label{rel:openflow}

Software-defined networking aims to provide programmable network functions,
network management, and forwarding logic to modern networks that need more
than conventional switches.
Much of the work on the development of technologies for SDN is targeted at or 
inspired by OpenFlow: a specification describing the general architecture of SDN 
switches and the protocol that allows them to be remotely observed and controlled 
\cite{openflow_spec}. The key parts of the OpenFlow switch architecture are a packet 
processing pipeline, an external controller which sends and receives
OpenFlow messages, and ports.

However, SDN standards like OpenFlow are not the limit of what programmable
network functions can become. OpenFlow artificially constrains the evolution
of SDN with their protocol-dependent model, messages, and switch structure.

Research into Steve and Freeflow used OpenFlow as a guide post for what an abstract
packet processing machine might look like and what language features may be
necessary to support such a machine. However, by no means did this research
strictly adhere to OpenFlow principles or semantics.
Steve and Freeflow, differ from OpenFlow concepts in a number of key ways.

Steve is protocol independent, whereas OpenFlow specifies a limited set of
support fields known as OXM fields.
Because Steve is protocol independent, its
pipeline model may contain a series of programmable decoders in between flow tables. 
OpenFlow does not say anything about packet decoding in its
specification as such behaviors are expected to be performed by specialized hardware
before pipeline processing. 
Steve does not currently support OpenFlow group tables.

The abstract SDN switch proposed by OpenFlow is not completely adopted by
the Freeflow runtime. Specifically, the remote OpenFlow controller was
removed from the Freeflow switch model. It is not desirable to remote configure
a switch for basic reactive programs. Sending messages to a controller through
a wire is a prohibitively high-latency operation.
Instead of communicating with a remote controller via messages, Freeflow uses an 
internal (non-distributed) controller which executes Steve event handlers. 
However, Steve is still capable of decoding OpenFlow messages like any other protocol
if there is a need for a distributed controller.

%An OpenFlow pipeline is defined as a sequence
%of one or more \textit{flow tables} and a \textit{group table} which handle
%packet lookups, decision making, and forwarding. Steve's packet processing pipeline differs from the traditional OpenFlow
%semantics in a number of ways. First,
%Steve provides language features for a user to create flow tables and define
%their flow entries, but does not yet support group tables.
%
%Second, a Steve pipeline has more than \textit{just} flow tables. Steve
%pipelines will also have \textit{decoders} which may be interleaved between
%tables. Decoders define how fields get decoded (or parsed) and extracted. In
%most switches, decoding is handled by specialized hardware that deal with
%well-known headers. OpenFlow only supports certain fields from these headers,
%known as OXM fields. However, the Freeflow data plane is
%protocol oblivious, knowing nothing about any specific headers; therefore
%decoding must be an explicit user-defined stage of the pipeline. By extension,
%Steve does not, by default, support these OXM fields either.

%Third, the semantics for packet handling using an external controller are different. An OpenFlow controller is software used to control an OpenFlow switch. It will handle ``exceptional events'' such as inserting,
%removing, and updating flow entries or processing packets which the pipeline
%could not. However, Steve and Flowpath do not use nor expect an external
%OpenFlow controller nor do they use OpenFlow messages to communicate with that
%controller.

%Steve and Flowpath attempt to reduce the role of the controller. Controller
%functions, which would typically handle exceptional cases (such as table
%manipulation and un-handled packets), can be written in Steve using special
%event handler functions. These event handlers are executed by the data plane
%on a special internal Flowpath ``controller'' thread rather than relying on an
%external controller.

\section{Early Steve Work}

Cassey, et al, provided much of the early Steve language work into programmatic
analysis of packet headers \cite{wripe}. Steve ensures the semantic constraints,
structural constraints (including buffer and view abstractions used by decoders), and 
safe access properties described by Cassey, et al. Steve also takes the idea
of full program analysis and applies it to pipeline stages to ensure requirement
constraints are not violated.

The Arbiter Framework was an early language implementation that allow for the
specification of data and control plane components. Arbiter used axiomatic 
programming concepts to ensure safe data path configuration by verifying that 
protocol definitions are not violated by the decoders which handle them
\cite{noproto1}.

\section{Languages for Programmable Data Planes} \label{rel:p4}

P4 is another protocol oblivious (independent) language for pipeline specification
 \cite{p4_spec, p4_spec2, p42014}.
P4 specifies a pipeline of parsers (equivalent to Steve decoders) followed
by a series of match+action
tables (equivalent to flow tables). Parsing state is stored in a data structure
called a ``parsed representation.''

Huawei's, Protocol-Oblivious Forwarding (POF) \cite{pof_fis, pof, pof_impl} provides a
low-level, assembly-like, instruction set for POF SDN switches. 
The POF programming model uses metadata as a ``scratch pad'' for parsing fields,
requiring that the programmer represent fields as generic \textit{\{offset, length\}} pairs, 
known as "search keys". POF switches also use a pipeline of tables, however,
a table extracts the fields it needs to match on from the metadata or packet buffer.
When performing table matching, a programmer specifies
an array of search keys needed by the table.

NetASM is an intermediate representation language for programmable data planes
\cite{shahbaz2015netasm}. It aims to solve the same issues as POF, but also provides a language that is target/device independent.

Steve's high-level pipeline specification syntax is similar to P4.
However, the underlying abstract pipeline model is closer to that of POF.
Steve decoders are placed between tables so that fields are only decoded as they are
needed similar to how POF tables extract their own fields.
Additionally, Steve supports partial decode of headers like POF to improve decoder
performance.
P4 fully parses all headers before table matching which Steve considers wasteful.

Steve uses a high-level language for pipeline specification because
the POF instruction set is difficult to write and
has no high-level language safety guarantees (such as a type and constraints system) 
layered on top of it.
POF's search keys require precise bit-offset and length values which are also too
difficult for a programmer to write.
Instead, Steve relies on the compiler to generate similarly efficient instructions
so that the programmer does not have to.


%Essentially, fields only get extracted as they are needed, right before matching
%against a given table.

%Steve is a language for modifying the packet processing and forwarding functionality of a data plane. Steve is \textit{protocol oblivious}, meaning it does not by default
%know of any well-known network protocols (IPv4, IPv6, TCP, etc). Instead, it
%provides language features for programmers to deal with any protocol, making the
%language more scalable for future protocols. Other SDN languages are pursuing
%these idea as well.
%
%The P4 language \cite{p4_spec, p4_spec2, p42014} is another high-level language for
%defining protocol oblivious packet processing pipelines. It is probably the
%most widely adopted SDN language. P4 allows users to define packet headers,
%packet parsers and
%match+action tables (which are equivalent to OpenFlow flow tables). Parsers are
%used to extract entire headers and store them in a "parsed representation"
%before entering pipeline processing. When the packet enters the pipeline,
%match+action tables match against fields in the parsed representation and
%perform actions on matched packets.
%
%Protocol-Oblivious Forwarding (POF) \cite{pof_fis, pof, pof_impl} is another
%project also tackling the problem of protocol oblivious pipeline processing. POF
%is a very low-level, assembly-like, instruction set. The POF instruction
%set gives the programmer very fine-grained control of which fields get parsed.
%The POF programming model uses metadata as a "scratch pad" for parsing fields,
%requiring that the
%programmer represent fields as generic \{offset, length\} pairs, known as
%"search keys". When performing table matching, a programmer specifies
%an array of these search keys to match against the flow table.
%Essentially, fields only get extracted as they are needed, right before matching
%against a given table.
%
%POF also supports additional actions that allow the data plane to manipulate
%flow tables. This includes adding, removing, and updating flow
%entries. Though these actions are traditionally left up to the controller, they
%are useful because they reduce the load on the controller and provide
%flexibility to the data plane. This feature is notably absent from other SDN
%languages.
%
%The POF instruction set is difficult to write in and does not have the safety
%guarantees of a higher level language. The programmer is burdened with the
%error-prone task of parsing by manually specifying which bits comprise a fields.
%A high level language compiling into POF instructions would thus be ideal.
%
%Steve tries to take the best of both worlds. Steve can define decoding functions
%similar to P4 parsers, with the added benefit of allowing the programmer to only
%extract the fields they need, reducing the amount of time needed to parse
%overall. The \{offset, length\} pairs describing each field get generated by the
%Steve compiler rather than being manually written by the programmer, thus
%reducing
%the risk for error. Steve decoders may also be interleaved between tables,
%meaning fields can be extracted, as needed, right before table matching like
%POF. Like P4,
%Steve decoding can also happen all at once "up-front," before any table
%matching.
%It is up to the programmer to decide which is better. This makes Steve
%decoders a little more robust than either P4 or POF.
%
%Steve supports high-level definition of flow tables (or match+action tables in
%P4). Additionally, Steve also allows the programmer to define flow entries. This
%means the match field values and actions for each flow entry can be expressed
%within the language, allowing the Steve compiler to ensure safety and
%correctness guarantees over them. For example, adding a flow entry that causes
%an infinite loop is always prevented. This task is normally left up to the
%controller during runtime, which can be slow. Steve also supports instructions
%for adding and removing these flow entries from tables like POF.
%
%Unlike P4 and POF, Steve is not \textit{just} an SDN specific language; it is an
%extension to a general purpose language. It thus supports features like
%functions, function calls, lexical scoping, loops, conditional statements, and
%local/global variables. Steve also has support for types, arithmetic
%expressions, and comparison expressions. On top of that, it automatically links
%against the C Runtime Library, supports calls to external functions, and may be
%statically or dynamically linked against any library.
%
%Admittedly, P4 is a more mature language than Steve. P4 supports things like
%meters, variable sized fields, table matching methods (Steve only supports exact
%matching), and certain actions that Steve does not. P4 can also target more
%platforms than Steve, which currently only compiles into modules for Freeflow.
%POF also supports certain actions and table matching methods that Steve does not
%currently support.

\section{SDN Controller Programming Languages} \label{rel:frenetic}

The Frenetic project has produced a family of network programming
languages. The Frenetic language is a declarative language, embedded in Python,
that uses SQL-like queries to classify packets and a library for describing
packet forwarding policies over a collection of network switches
\cite{foster2011frenetic, foster2013frenetic}. This language is designed to
abstract away the difficulties of programming a centralized SDN controller. Its
sister project, Pyretic, does the same with some divergence in how forwarding
policies are expressed \cite{modularpyretic}.

NetCore is a network programming language for expressing forwarding policies and
generating classifiers from those policies \cite{monsanto2012netcore}. These
classifiers then get installed on switches. NetKAT is a language, inspired by
NetCore, for expressing packet processing functions and reasoning about a
network using Kleene Algebra \cite{kozen2014netkat, anderson2014netkat}. NetKAT
attempts to prove the correctness of its network programs through mathematical theory.

\section{Packet Parsers and Header Specifications}

Some early DSLs focused on binary header specification languages that are similar to 
those used by Steve and P4 \cite{binpac, packet_types, datascript}.

Gibb, et al. described the design principles of different kinds of packet parsers 
\cite{parser2013gibb}. Steve decoders fit into the non-streaming programmable parsers 
described in this paper. 

The Kangaroo packet parser design tries to solve the problem of designing high performance and flexible packet parsers \cite{kangaroo}. Kangaroo can "compile" parser instructions from parse graphs. Kangaroo can analyze these graphs and generate "lookahead" parsers which can parse several headers at once.

The BSD packet filter (BPF) supports a low-level instruction set for parsing packets and defining packet filtering rules \cite{bpf1993mccanne}. It is most commonly used for network monitoring tools.

\section{Data Plane Programming Libraries}
\label{rel:odp}

Data Plane Development Kit (DPDK) by Intel is a set of software libraries for improving packet processing speeds on Intel processors and NICs \cite{dpdk_webpage}. DPDK attempts to provide a single platform (Intel processors) for performing all packet processing tasks, thus eliminating the need for specialized hardware.

Open Data Plane (ODP) is an open source API in C for developing data plane applications \cite{odp_webpage}. ODP provides application portability by providing a common set of APIs across multiple platforms and instruction set architectures such as ARM, MIPS, x86, and proprietary hardware.

\section{Open VSwitch}
\label{rel:vswitch}

Open vSwitch is an SDN virtual, software switch which supports programmatic extensions \cite{ovs_webpage, ovs2009extending, ovs2013}. Open vSwitch was designed to work in virtual environments and runs on the hypervisor. It provides a high-level command line interface for re-configuring switches \cite{ovs_man_page}.

\section{Freeflow}
\label{rel:freeflow}

Freeflow is an SDN software switch architecture developed
by Flowgrammable \cite{freeflow_software}. The Freeflow virtual machine (FFVM) 
provides a programmable, protocol oblivious, software data plane that 
supports multi-tenancy. Steve is
designed to generate code targeting the FFVM runtime environment. A Freeflow data 
plane loads Steve applications which control the logic behind its packet processing 
pipeline.
